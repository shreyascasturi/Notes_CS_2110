\documentclass[12pt]{article}
\usepackage{amsmath}
\usepackage[left=1in, right=1in, top=1in, bottom=1in]{geometry}
\usepackage{amsthm}
\usepackage{array,mathtools}
\newcommand*{\carry}[1][1]{\overset{#1}}
\newcolumntype{B}[1]{r*{#1}{@{\,}r}}
\newtheorem*{remark}
{Remark}
\title{Notes on CS 2110}
\author{Shreyas Casturi}
\begin{document}
\maketitle

\newpage

\section*{Preface}
These notes are a way for me to prove I understood the 2110 material.  They are intended as a reference.

\newpage

\section*{Chapter 1: Datatypes}

If we aim to have any sort of discussion about the architectures that underpin computers, we must first
understand \textit{binary} numbers and their importance. Indeed, a base 10 numerical system does not lend itself well to computing.

When I say, "what is five?", you may draw the number five, the word "five", five lines, a mathematical expression that evaluates to five, so on and so forth.
What we notice is that these are simply \textit{ways or schemes} of denoting something.

So, our base-10 numeric system is another \textit{scheme} of denoting quantities that have no real name. After all, numbers exist independently of us.

Might there be an easier way of expressing what we already know?

\subsection*{Binary Numbers}

In the early days of computing, much detail was devoted to the electronics and hardware. Scientists realized that circuits
obeyed a simple rule -- either they were on, or they were off. There was no inbetween.

This made it easier to accept the \textit{base 2 numeric system}. Numbers expressed in this system are
known as \textit{binary} numbers. A single element
of a binary number is known as a \textit{binary digit}, or a \textit{bit}. A binary
digit has one of two values: 0, or 1. A binary number
is composed of \textit{only} binary digits.

So, a random binary number might look like this

\[number = 010001010101000100111000010101\]

Given $n$ bits, how many possible binary numbers can I generate?

\begin{remark}
    Given $n$ binary digits, there are $2^n$ possible binary numbers.
\end{remark}

\begin{proof}
    There are 2 numbers possible for each of the $n$ digits. Multiply
    this out to obtain the result. Start with 1 or 2 digits, and work up from there.
\end{proof}


    Of course, these numbers are worthless on their own. If someone told you that there were "0101010101"
    cats stuck in that burning house on the left, you'd probably look at them funny. You might even inhale some smoke
    and leave. Those poor cats are dead.

    \subsection*{Assignations in Binary}

    For four binary digits, 16 combinations are generated.
    We may naturally try to assign our old numbers (which we'll call integers) to these 16 combinations.
    But how many orderings are possible?

    There are $n!$ orderings possible, and we'd like to create an ordering
    that is easy enough for all of us to follow. Indeed, we could
    just do something like this:

    \newpage

    \begin{center}
        \begin{tabular}{|c|c|}
          \hline
          Integers & Binary numbers \\
          \hline
          0 & 0001 \\
          1 & 1010 \\
          2 & 1101 \\
          $\vdots$ & $\vdots$\\
          \hline
                     
        \end{tabular}
    \end{center}
    This really isn't helpful, is it.

    Well, what happens if we assign 0000 to 0, naturally, and 1111
    to be the largest possible integer that could be generated? For 16 possible
    combinations, obviously the range is from $[0 \rightarrow 15]$, or
    to put it a little more mathematically, $[0 \rightarrow 2^n - 1]$.

    The most interesting part is this: Let us have only two binary digits. There
    are then four possible binary numbers, and four integers
    we could assign. If 00 is 0, then 11 is 3. Let 01 be 1, and let
    10 be 2.

    Then we get this chart:

    \begin{center}
        \begin{tabular}{|c|c|}
          \hline
          Integers & Binary numbers \\
          \hline
          0 & 00 \\
          1 & 01 \\
          2 & 10 \\
          3 & 11 \\         
          \hline
                     
        \end{tabular}
    \end{center}

    Observe the progression of the binary numbers. Each time we add 1 to our integers, digits
    are being flipped, but exactly what digits?

    Let us do this one last time for 3 binary digits. How many binary numbers can we generate?
    And what is the range of the integers for these binary numbers?

    The chart follows.

        \begin{center}
        \begin{tabular}{|c|c|}
          \hline
          Real numbers & Binary numbers \\
          \hline
          0 & 000 \\
          1 & 001 \\
          2 & 010  \\
          3 & 011 \\
          4 & 100 \\
          5 & 101 \\
          6 & 110 \\
          7 & 111 \\
          \hline
                     
        \end{tabular}
    \end{center}

    A relationship is obvious from the numbers generated, but may not be intuitive.
    Here is what we know, so far, about binary numbers:

    \newpage

    \begin{enumerate}
        \item A binary number is formed from binary digits.
        \item Each binary digit can be 0 or 1.
        \item There are $2^n$ binary numbers generated from $n$ digits.
        \item We know that we could assign $2^n$ of our integers
        to the $2^n$ binary numbers we generate from $n$ binary digits.
        \item We know there are certain intuitive ways to begin this ordering.
        \item Let the $n$ binary digits be all 0 to denote 0, and let the $n$ digits be all 1 to denote the largest possible integer.
        \item But... what happens after that?
    \end{enumerate}

    The more general question we might ask ourselves is \textit{how do we actually determine what numbers to assign to what combinations}?

    One way to do this is to view \textit{each} binary digit as a power of 2. Let us do this like so, for four binary digits:

    \begin{center}
        \begin{tabular}{|c|c|c|c|c|c|}
          \hline
          Integers & $2^3 = 8$ & $2^2 = 4$ & $2^1 = 2$ & $2^0 = 1$ & Binary numbers \\
          \hline
          0 & 0 & 0 & 0 & 0 & 0000 \\
          1 & 0 & 0 & 0 & 1 & 0001 \\
          2 & 0 & 0 & 1 & 0 & 0010 \\
          3 & 0 & 0 & 1 & 1 & 0011 \\
          4 & 0 & 1 & 0 & 0 & 0100 \\
          5 & 0 & 1 & 0 & 1 & 0101 \\
          6 & 0 & 1 & 1 & 0 & 0110 \\         
          7 & 0 & 1 & 1 & 1 & 0111 \\
          8 & 1 & 0 & 0 & 0 & 1000 \\
          9 & 1 & 0 & 0 & 1 & 1001 \\
          10 & 1 & 0 & 1 & 0 & 1010 \\
          11 & 1 & 0 & 1 & 1 & 1011 \\
          12 & 1 & 1 & 0 & 0 & 1100 \\
          13 & 1 & 1 & 0 & 1 & 1101 \\
          14 & 1 & 1 & 1 & 0 & 1110 \\
          15 & 1 & 1 & 1 & 1 & 1111 \\
          \hline
        \end{tabular}
    \end{center}

    Then, we can intuitively break down any binary number given to us by writing
    the powers of 2 above these numbers, starting with $2^0$ for the right-most digit,
    and ending with $2^{n - 1}$ for the $n$ digits given to us.

    Notice then that all of these numbers can be written as sums of various powers of 2.

    Let us ask ourselves some questions:

    \begin{enumerate}
        \item Given $n$ binary digits, how many possible binary numbers can be made?
        \item Given $n$ binary digits, what is the largest power of 2 possible, and what digit
        is it found at?
        \item Given $n$ binary digits, how many possible integers can be assigned, and what
        is the range of these integers, assuming only 0 and the positive integers are used?
    \end{enumerate}
    

    \begin{remark} Some Facts On Binary Digits:
            \begin{enumerate}
                \item For $n$ binary digits, $2^n$ binary numbers can be generated.
                \item Following from the chart above, the largest power of 2 possible is
                $2^{n - 1}$. It is found at the left-most digit.
                \item Given that we map integers to binary numbers in a one-to-one correspondence,
                then $2^n$ integers can be assigned. Assuming only 0 and the positive integers are used,
                then the range of integers is $[0 \rightarrow (2^{n} - 1)]$.
            \end{enumerate}
        \end{remark}

        %%But this all begs the question: \textit{what about negative numbers?}

        
Before we do anything else with binary numbers, we should first make sure we are comfortable with
basic operations involving those binary numbers.

\subsection*{Addition and Subtraction In Binary}

We should get comfortable thinking in binary terms, so this means we should have some experience
operating with binary numbers -- namely in addition and subtraction. We will use the binary-integer
relationships we have described above to help us figure out what our binary result should be.

We have already discussed that there are only ever 2 values in each bit of a binary number -- 0 or 1, and that
the place values are simply powers of 2, starting from $2^0$, all the way to $2^{n - 1}$.

With this in mind, we shall pose some problems.

\subsubsection*{Basic Addition}

\textbf{Problem 1: } Let $A = 0010 \; 1110$, and let $B = 0110 \; 1101$. What is $A + B$? \\

\textbf{Answer: }

We can write the problem as

\begin{equation*}
    \begin{array}{B3}
      0010 &              1110 \\ 
      {}+ 0110 &              1101 \\ \hline
          xxxx & xxxx
    \end{array}
    \end{equation*}

    \begin{enumerate}
        \item (Both Zero Bits): $0 + 0 = 0$
        \item (Bits Not Equal): $0 + 1 = 1 + 0 = 1$.
        \item (Both Bits == 1): $1 + 1 = 0 (\text{carry over 1})$
    \end{enumerate}
    
    The rule for addition is simple -- if both bits are 0, the answer bit is 0. If both of the
    bits are not the same, then it's a 1, with no carry. If both of the bits are 1, then the result
    is a 0, and you carry a 1 over. Basically, 1 + 1 = 0 (with a carry of 1 to the next digit).

    So, our addition problem then becomes

\begin{equation*}
    \begin{array}{B3}

      \carry0 \carry0 1 \carry 0 &              \carry 1 110 \\ 
      {} + 0110 &              1101 \\ \hline
           1001 &                  1011
                                 
    \end{array}
    \end{equation*}

    What is this answer in base 10 format?

    I won't write out the whole tables, but
    you get

    \[1001 \; 1011_2 = 2^7 + 2^4 + 2^3 + 2^1 + 2^0 = 128 + 16 + 8 + 2 + 1 = 155. \]

    If you turn $A$ and $B$ into their base-10 expressions and add those two integers together,
    you should get the same result. If you get stuck, try turning your adding terms into
    their base-10 forms and adding from there. It is a handy reference.

    \subsubsection*{Basic Subtraction}
    This will be filled in later.



    Now that we know how to work with binary numbers and perform operations on them,
    we may ask ourselves a simple question -- we've only covered the positive integers. \textit{What about negative
    numbers?}



  \subsection*{Beyond The Positive}
  We usually deal with both positive and negative integers, but so far we've only dealt with positive integers
  and zero, strictly. We must find a way to incorporate negative integers.

  
  \subsubsection*{Signed Magnitude}
  
  One logical idea is that given a binary number of $n$ bits, then reserve one bit -- the leftmost bit,
  or the most significant bit -- to represent the sign
  of the number. This is called the \textit{signed magnitude}.

  So 0000 is 0, 1000 is -0, 0001 is 1, and 1001 is -1, so on and so forth.

  We will make use of a chart from the previous sections and continue to update as such, with
  each scheme we get/invent.

  \textbf{We shall use this chart to show four-bit numbers and our various numbering schemes.}

  The first two are the unsigned integers, and signed magnitude. 
    \begin{center}
        \begin{tabular}{|c|c|c|c|c|}
          \hline
          Binary Numbers & Unsigned Integers & Signed Ints \\
          \hline
          0000 & 0 & 0\\ 
          0001 & 1 & 1\\  
          0010 & 2 & 2\\  
          0011 & 3 & 3\\  
          0100 & 4 & 4\\  
          0101 & 5 & 5\\  
          0110 & 6 & 6\\         
          0111 & 7 & 7\\  
          1000 & 8 & -0\\  
          1001 & 9 & -1\\  
          1010 & 10 & -2\\
          1011 & 11 & -3\\
          1100 & 12 & -4\\
          1101 & 13 & -5\\
          1110 & 14 & -6\\
          1111 & 15 & -7\\
          \hline
        \end{tabular}
    \end{center}
  
        
\end{document}

