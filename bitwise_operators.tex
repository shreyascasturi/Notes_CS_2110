\documentclass[12pt]{article}
\usepackage{amsmath}
\usepackage{setspace}
\usepackage[left=1in, right=1in, top=1in, bottom=1in]{geometry}
\usepackage{amsthm}
\usepackage{array,mathtools}
\newcommand*{\carry}[1][1]{\overset{#1}}
\newcolumntype{B}[1]{r*{#1}{@{\,}r}}
\newtheorem*{remark}
{Remark}
\title{Bitwise Operations -- Chapter 2, For CS 2110}
\author{Shreyas Casturi}
\date{}
\begin{document}
\maketitle

\newpage
\doublespacing
\tableofcontents
\singlespacing

\newpage

\section{Chapter 2: Bitwise Operations}

Now that we have an understanding of how to work
with basic binary numbers, we can now learn about
\textit{logical operators} that work on
bits/binary numbers, such as $AND, OR, NOT$. You may have seen these operators
before.

We will deal with the concept of \textit{floating-point numbers},
which carry more precision, but are conceptually harder
to represent.

We will also learn to work with other numerical
systems, such as octal and hexadecimal, with examples of conversion between these systems.

\subsection{Logical Operators}

Logical operators are done on bits, and are hence called \textit{bitwise} operators.

We will denote a sub-sub-section to each major
logical operator, usually showing two truth tables. Exercises
will give you actual experience.

\subsubsection{AND}

A bitwise AND operation works like so,
for given bits 0 and 1:

\begin{center}
    
    \begin{tabular}{|c|c|c|}
      \hline
    AND & 0 & 1 \\
    \hline
      0 & 0 & 0 \\
      \hline
    1 & 0 & 1 \\
    \hline
    \end{tabular}
   
\end{center}

A more general example is

\begin{center}
    \begin{tabular}{|c|c|c|}
      \hline
      A & B & (A AND B, AB, $A \; \& \; B$) \\
      \hline
      0 & 0 & 0 \\
      \hline
      0 & 1 & 0 \\
      \hline
      1 & 0 & 0 \\
      \hline
      1 & 1 & 1 \\
      \hline
    \end{tabular}
    
\end{center}
\newpage
\subsubsection{OR}

An OR operation works as so

\begin{center}
    \begin{tabular}{|c|c|c|}
      \hline
      OR & 0 & 1 \\
      \hline
      0 & 0 & 1 \\
      \hline
      1 & 1 & 1 \\
      \hline
      
    \end{tabular}
\end{center}

A general example is

\begin{center}
    \begin{tabular}{|c|c|c|}
      \hline
      A & B & (A OR B, A + B, $A \; | \; B$) \\
      \hline
      0 & 0 & 0 \\
      \hline
      0 & 1 & 1 \\
      \hline
      1 & 0 & 1 \\
      \hline
      1 & 1 & 1 \\
      \hline
    \end{tabular}
\end{center}

There is a difference between this: $||$, and $|$. The latter is the bitwise OR
operator, which does operations on bits, but the former is a conditional OR,
used in evaluating statements, as seen in Java, C, and other languages.
\subsubsection{NOT}

A NOT operation doesn't require two bits, but rather one. A NOT operation
negates/flips the present value.

So, we have

\begin{center}
    \begin{tabular}{|c|c|}
      \hline
      NOT & Result \\
      \hline
      0 & 1 \\
      \hline
      1 & 0 \\
      \hline
    \end{tabular}
\end{center}

A NOT can be represented as a ``~".
     
\subsubsection{XOR}

An XOR operation is more interesting, and does require two bits/binary numbers.
If the bits are the same, then we return 0, but if the bits aren't the same, we return 1.

We obtain

\begin{center}
    \begin{tabular}{|c|c|c|}
      \hline
      XOR & 0 & 1 \\
      \hline
      0 & 0 & 1 \\
      \hline
      1 & 1 & 0 \\
      \hline
    \end{tabular}
\end{center}

A general operation can be seen as

\begin{center}
    \begin{tabular}{|c|c|c|}
      \hline
      A & B & A XOR B\\
      \hline
      0 & 0 & 0 \\
      \hline
      0 & 1 & 1 \\
      \hline
      1 & 0 & 1 \\
      \hline
      1 & 1 & 0 \\
      \hline
    \end{tabular}
\end{center}

\subsubsection{NAND}

A NAND can be represented as a negation of the result of the AND operation.
So, the result of A NAND B is equivalent to !(A \& B).
\begin{center}
    \begin{tabular}{|c|c|c|}
      \hline
      NAND & 0 & 1 \\
      \hline
      0 & 1 & 1 \\
      \hline
      1 & 1 & 0 \\
      \hline
    \end{tabular}
    \end{center}

    Generally, this is written as

    \begin{center}
        \begin{tabular}{|c|c|c|}
          \hline
          A & B & A NAND B \\
          \hline
          0 & 0 & 1 \\
          \hline
          0 & 1 & 1 \\
          \hline
          1 & 0 & 1 \\
          \hline
          1 & 1 & 0 \\
          \hline
        \end{tabular}
    \end{center}
    
\subsubsection{NOR}


\subsection{Bit Vectors}


\subsection{Hexadecmial}

\subsubsection{Converting from Binary to Hexadecimal, Vice Versa}

\subsection{Octal}

\subsubsection{Converting from Binary to Octal, Vice Versa}

\subsection{Floating Point Numbers}



\end{document}